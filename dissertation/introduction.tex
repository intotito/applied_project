\chapter{Introduction}

\textbf{F}irst-person shooter games represent the class of games where the player views the environment through a viewport
and can perform such actions as looking around, moving around, aiming, and firing of weapons. these actions are accomplished
using various buttons or combinations of buttons.

During a typical gameplay, players are confronted with other opposing players and are required to eliminate their opponents
using various weapons available while evading enemy fire. To successfully compete in such a scenario, players are expected to
react fast, effectively track targets, accurately hit targets, perceive sound, and accurately map them to a location within
their environ.

The goal of this research work is to collect biometric data from users using a Wearable Smart Watch (Polar Vantage V2)
and performance data using a Test Game designed in a first-person-shooter scenario for this purpose. This involved building
the required infrastructure to reliably collect and store users' data from the various sources. Finally, the data collected will be
analyzed using classical machine learning algorithms and automatic neural network to ascertain if a correlation exists between
users' performance in a digital game scenario and their biometric data.


\section{Background}
PUBG: Battlegrounds (previously known as PlayerUnknown\'s Backgrounds) is a battle royale style player versus
player (PvP) shooter game developed by PUBG Studio. Players face-off with each other using various types of battlefield weapons
in a last-man-standing deathmatch and the last person to remain alive wins. The game is available on all major platforms
and as of March 2021, the mobile version of the game has accumulated more than a billion download outside of China with
revenue of over \$9billion while the PC and console versions have accumulated a total revenue of \$4billion
\cite{statista}.
\par
Since its first release in 2017, the game has since become one of the fans favourite and has over `350,000' peak concurrent
monthly users~\footnote{statista}. As a multiple award-winning game with proven longevity records and a large community.
Interest in the game cuts across different demography and is equally far-reaching across the globe.
The game-playing scenario requires players to face off with other players and there is where some skills like
`eye-hand-coordination', `ear-hand-coordination', `fine-motor' skills, etc\.. are required to compete favourably against
other players. Players have access to a variety of weapons with different capabilities and can make in-game adjustments
to their control to suit their various preferences.

This project is a continuation of research work previously done by Fourth Year Software Design Students titled `Biometric
Data Collection for Performance Optimization in a Digital Game Scenario' in collaboration with the Department of Sports
\& Exercise Science, Atlantic Technological University.
The originating project titled \textbf{`Biometric Data Collection for Performance Optimization in a Digital 
Game Scenario'}. A desktop application Test Game designed with Unity 3D Framework and a Firestore database
that stores all user data were part of the infrastructure inherited from the previous project.

\section{Objectives}
This project aims to develop a robust and enduring system that can analyze users' data on the fly using a predefined machine learning
algorithm. The system is expected to function well beyond the scope of this research and is expected to build better models as more data
is collected. The data collection effort involved enrolling volunteers to gather biometric and test data. Each volunteers were given
a Polar Vantage V2 Smart Watch and a Test Game which they were expected to play daily.

\subsection{Volunteer Recruitment}
Volunteers were sought to aid the data collection effort. At the onset of the project, a total of 10 volunteers were expected to be recruited
to meet the target of 1000 data points set out as the goal to build a machine learning model. The recruitment process involved getting
ethics approval from the institution and the volunteers were expected to sign a consent form before participating in the research. 
Flyer posters were designed and distributed across the campus to attract potential volunteers. A total of 5 volunteers were eventually 
recruited and participated in the research. 

\subsection{Feature Selection}
The features selected for this research were based on the availability of data from the wearable device and the relevance of the
features to the research. The features were divided into two categories: Independent and Dependent variables. The independent
variables are the biometric data collected from the wearable device, while the dependent variables are the performance metrics


\subsubsection{Performance Metrics (Dependent Variables)}
For the purpose of measuring users' performance in in a first-person shooter game scenario, three categories of metrics were
developed to measure users' performance. They are listed as follows: Fine-Motor Control, Visual metrics, and Audio metrics.

\textbf{Fine Motor Control} Fine motor refers to the controlled and coordinated incremental movements made by the hand
when handling items. This metric assesses how quickly users are able to adjust their aim to targets appearing on the screen.
Two take away from this classification are the Response Time (seconds) which quantifies reaction time and Accuracy (\%) which
quantifies users' fine adjustment capabilities.

\textbf{Visual Reflexes} The game scenario under study for most of the time involves users having multiple legitimate targets
to shoot, and having to engage them simultaneously. This metric accesses users' visual reflexes by considering the Target
Accuracy which is the measure of the number of successful targets hit with the total number of targets spawned expressed in
percentage. Shot Accuracy which is the measure of the number of successful shots to the number of targets hit. Finally, the
Average Response Time which is the average time it takes for a target to appear before being hit.

\textbf{Audio Reflexes} User audio reflexes are accessed using the Response Time, which measures the time it takes to
identify the source of a sound within the user's environment.

\subsubsection{Biometric Data (Independent Variables)}
The collection and analysis of biometric data play a significant role in understanding the relationship between physical fitness and
gaming performance. This research aims to investigate how various physical parameters, such as HRV (Heart Rate Variability), Heart
Rate Max, Heart Rate average, Active Sleep and Quality of Sleep, influence a user's performance in a digital game, in this case
PUBG: Battlegrounds. The research will also explore the potential of using biometric data to suggest the most suitable settings for
different game scenarios. By using wearable technology, which can monitor metrics such as heart rate, sleep, and activity, this study
seeks to establish a correlation between the player's physical condition and their performance in a digital game scenario. The
importance of fitness data in this context cannot be overstated. Heart Rate Variability (HRV), maximum heart rate, active sleep, and
sleep quality are all essential indicators of a person's physical condition. These metrics offer a comprehensive insight into the
user's physical condition, which can be used to improve their gaming performance.

\textbf{Heart Rate Variability (HRV):} is a measure of the variation in time between each heartbeat, and it is closely linked to
the body's stress levels. HRV is widely used as a measure of the body's autonomic nervous system, which controls the body's stress
response. A higher HRV is associated with a lower stress level, while a lower HRV is associated with a higher stress level. A research
from E. Ortega\cite{ortega2018pre} highlights the importance of HRV in sports science, as a way of understanding the psychological
state of athletes before completion. It found a positive correlation between HRV, self-efficacy, and performance among sport shooters.
Advanced shooters demonstrated lower average heart rates and employed mental skills more effectively than less experienced shooters.
This suggests HRV as a valuable asset when transferring the psychological state of the athlete to the performance in a digital game.
It allows for a better understanding of the player's stress levels, and how it affects their performance. This personalized approach
can enhance player experience, and potentially improve their performance in the game.

The Heart Rate Variability measurements typically requires a chest strap, and measurements typically span from 5 minutes to 24 hours,
it is commonly used in clinical settings to evaluate cardiac conditions.\cite{malik1996heart} Short-term HRV analysis lasting less then
5 minutes have been proven to provide more accurate estimations compared to longer measurements.\cite{mcnames2006reliability}
Based on this, the research will focus on short-term HRV measurements, as it is more practical for the user, and it provides accurate
estimations. It will be more detailed in the methodology section.

\textbf{Heart Rate Max:} is the maximum number of times the heart can beat in a minute, and it is a measure of the body's
cardiovascular fitness. It is an important indicator of the body's physical condition, and it is used to evaluate the body's ability
to perform physical activities, reflecting their cardiovascular fitness. In this research, understanding HRmax is vital for assessing
participant's fitness and endurance. It could indicate how well players can handle stress and maintain their concentration over
extended periods of time. By analysing alongside other biometric data this research could uncover valuable insight into optimizing
players performance in a digital game scenario.

\textbf{Heart Rate Average:} is the average number of times the heart beats in a minute, and it is a measure of the body's physical
activity. It is also an important indicator of the body's physical condition, and it is another measure of the body's
cardiovascular fitness. It is used to evaluate the body's ability to perform physical activities, reflecting their cardiovascular
fitness.

\textbf{Active Step:} Active steps, as a measure of physical activity, is an important indicator of the body's physical condition. It is
used to evaluate the body's ability to perform physical activities, reflecting their cardiovascular fitness. Counting the number of
steps taken daily can help track overall physical activity, and it is a good indicator of the body's physical condition. 


\textbf{Quality of Sleep:} are important indicators of the body's recovery and readiness for physical activity. Active sleep is a
measure of the body's physical activity during sleep, and it is an important indicator of the body's recovery and readiness for
physical activity. High-quality sleep, marked by sufficient duration and minimal disruptions, is fundamental for overall health
impacting mood, and cognitive function. A research that evaluated Sleep and performance in Eathletes \cite{BONNAR2019647} underscores the
critical role of sleep in Eathletes performance in Esports, showing its impact on cognitive functions as a crucial factor for
competitive success. Adequate sleep improves information processing, visual motor functioning, attention, working memory, and other
functions essential for decision-making and reaction time. On the other hand, sleep deprivation can significantly impair these
cognitive abilities, potentially to poor performance. For this research it proves the importance of sleep in the context of gaming
performance, and how it can be used to predict the player's readiness and tune their performance in a digital game scenario.




\subsection{Data Analysis}
Some selected features were dropped before analysis due to the unavailability of data and unexpected configuration of the wearable device. The Active Steps and
Maximum heart rate were dropped from the independent variable features because of an unexpected configuration of the wearable device required the volunteers
to manually enable some features on the device before data could be collected.
A total of 440 data points were collected from the volunteers at the time of reporting and 92 of these data points representing 21\% were deemed valid. 
The validity of data points was determined by the availability of data from the wearable device and the test game within the same time frame under 
consideration. 
Machine Learning models were designed in a pipeline to clean and normalize the data before the commencement of training and other relevant analyses. 
Cleaning involves removing missing values, normalizing involves scaling the data to a range of 0 to 1 to ensure that no feature dominates and potentially 
adding bias to the model. This is particularly useful as a good few of the variables were recorded in varying units. As part of the best model selection 
algorithm incorporated into the pipeline, the data is further scaled to a mean of 0 and a standard deviation of 1 for a potentially better model.

A correlation matrix was generated using the Pearson correlation coefficient algorithm to ascertain the relationship between the independent and dependent 
variables. This provided an insight into the data and certified the independence of the independent variables from each other and on the other hand, shows
some level of correlation between the independent and dependent variables. These correlations were further corroborated by the machine learning models 
developed in the research to map independent variables to a dependent variable.

The best model selection was incorporated into the algorithm using the GridSearchCV algorithm from the sklearn library. It involved training various models with
a suite of M/L Algorithms (Random Forest, KNN, SVM, Logistic Regression) using various hyperparameters to determine the best model for the data. Best Model
selection decision was based on a 5-fold cross-validation technique using a validation dataset. This technique was applied to the various dependent variables
to get the best performant model. An Automatic Neural Network was incorporated into the pipeline to validate the results from the classical model. This 
was done to provide a reference point to ascertain if there is likely overfitting or underfitting in the models. The Automatic Neural Network was trained
using the same data as the classical model and the results were compared to determine if both models follow the same trend. 

\subsection{Infrastructure}
The Amazon Web Services (AWS) was used to host the web application. Various services were incorporated into the system to provide all the functionalities
stated. The AWS RDS configured with MySQL was used to store data in an organized form, AWS EC2 was used to provide compute time and storage for the web
application, Amazon Certificate Manager used for SSL certification, Amazon Elastic Load Balancer for load balancing and Amazon Route 53 used for domain name. 

