\chapter{Introduction}

\textbf{F}irst person shooter games represents the class of games where the  player views the environment through a viewport
    and can perform such actions as looking around, moving around, aiming and firing of weapons. these actions are accomplished
    using various button or combination of button.

    During a typical gameplay, players are confronted with other opposing players and are required to eliminate their opponents 
    using various weapons available while evading enemy fire. To successfully compete in such a scenario, players are expected to 
    react fast, effectively track targets, accurately hit targets, perceive sound and accurately map them to a location within 
    their environ.
    
    The primary objective of this research work is to formulate a metrics that can accurately measure users performance in the 
    such scenario and compare their performance with their fitness data with the aim of finding a correlation between 
    performance and physiological state.
    
    \section{Background}
    PUBG: Battlegrounds (previously known as PlayerUnknown\'s Backgrounds) is a battle royale style player versus
    player (PvP) shooter game developed by PUBG Studio. Players face-off with each other using various types of battlefield weapons
    in a last man standing deathmatch and the last person to remain alive wins. The game is available in all major platforms
    and as of March 2021, the mobile version of the game has accumulated more than a billion download outside of China with 
    revenue of over \$9billion while the PC and console versions have accumulated a total revenue of \$4billion
    \cite{statista}.
    \par 
    Since its first release in 2017, the game has since become one the fans favorite and has over `350,000' peak concurrent 
    monthly users~\footnote{statista}. As a multiple award-winning game with proven longevity records and a large community.
    Interest in the game cut across diferent demography and is equally far-reaching across the globe. 
    The game playing scenario requires players to face-off with other players and there is where some skills like 
    `eye-hand-coordination', `ear-hand-coordination', `fine-motor' skills, etc\.. are required to compete favorably against 
    other players. Players have access to a varieties of weapons with different capabilities and can make in-game adjustments
    to their control to suite their various preferences.

    This project is a continuation of research work previously done by Fourth Year Software Design Students titled `Biometric 
    Data Collection for Performance Optimization in a Digital Game Scenario' in collaboration with the Department of Sports
     \& Excercise Science, Atlantic Technological University.
    The originiating project titled \textbf{`Biometric Data Collection for Performance Optimization in a Digital
    Game Scenario'}, posed the question `can a player\'s biometric data be used to optimize their performance in a 
    first-person shooter game'? And a subsequent follow up project which sought to create a Chart API capable of displaying 
    all relevant information previously displayed on different pages on a single page.\\
    The former research was geared towards creating a platform for collecting performance data in a similar scenarios 
    (Weapons, controls, user perspective, ect\..) obtainable in PUBG\::Battlegrounds in the form of a Unity Desktop 
    {\tt Application}. Collection and storage of Biometric data from an 
    {\tt Activity Monitor} in the form of a {\tt Smart Watch}. With the eventual goal of finding correlation between their
    performance and their Biometric data. 


    \section{Performance Metrics}
    For the purpose of measuring users performance in in a first-person shooter game scenario, three categories of metrics where
    developed to measure users performance. They are listed as follows: Fine-Motor Control, Visual metrics and Audio metrics. 
    \subsection{Fine Motor Control}  Fine motor refers to the controlled and coordinated incremental movements made by the hand
    when handling items. This metrics assesses how quickly users are able to adjust their aim to targets appearing on the screen.
    Two take away from this classification is the Response Time (seconds) which quantifies reaction time and Accuracy (\%) which 
    quantifies users fine adjustment capabilities. 
    \subsection{Visual Reflexes} The game scenario under study for most of the time involves users having multiple legitimate targets
    to shoot, and having to engage them simultaneously. This metrics accesses users visual reflexes by considering the Target 
    Accuracy which is the measure of the number of successful targets hit with the total number of target spawned expressed in 
    percentage. Shot Accuracy which is the measure of the number of successful shots to the number of targets hit. Finally the 
    Average Response Time which is the average time it takes for a target to appear before being hit.
    \subsection{Audio Reflexes} User audio reflexes are accessed using the Response Time, which measures the time it takes to 
    identify the source of a sound within the users environment
    
    \section{Biometric Data} 
    The collection and analysis of biometric data play a significant role in understanding the relationship between physical fitness and 
    gaming performance. This research aims to investigate how various physical parameters, such as HRV (Heart Rate Variability), Heart 
    Rate Max, Heart Rate average, Active Sleep and Quality of Sleep, influence a user's performance in a digital game, in this case 
    PUBG: Battlegrounds. The  research will also explore the potential of using biometric data to suggest the most suitable settings for
    different game scenarios. By using wearable technology, which can monitor metrics such as heart rate, sleep, and activity, this study
    seeks to establish a correlation between the player's physical condition and their performance in a digital game scenario. The 
    importance of fitness data in this context cannot be overstated. Heart Rate Variability (HRV), maximum heart rate, active sleep, and 
    sleep quality are all essential indicators of a person's physical condition. THese metrics offers a comprehensive insight into the 
    user's physical condition, which can be used to improve their gaming performance. 

    \subsection{Heart Rate Variability (HRV):} is a measure of the variation in time between each heartbeat, and it is closely linked to 
    the body's stress levels. HRV is widely used as a measure of the body's autonomic nervous system, which controls the body's stress 
    response. A higher HRV is associated with a lower stress level, while a lower HRV is associated with a higher stress level. A research
    from E. Ortega\cite{ortega2018pre} highlights the importance of HRV in sports science, as a way of understanding the psychological
    state of atheletes before competions. It found a positive correlation between HRV, self-efficacy, and performance among sport shooters.
    Advanced shooters demonstrated lower average heart rates and employed mental skills more effectively than less experienced shooters.
    This suggests HRV as a valuable asset, when tranferring the psychological state of the athlete to the performance in a digital game. 
    It allows for a better understanding of the player's stress levels, and how it affects their performance. This personalized approach 
    can enhance player experience, and potentially improve their performance in the game. 
    
    The Heart Rate Variability measurements typically requires a chest strap, and measurements typically span from 5 minutes to 24 hours, 
    it is commonly used in clinical settings to evaluate cardiac conditions.\cite{malik1996heart} Short-term HRV analysis lasting less then 
    5 minutes have been proven to provide more accurate estimations compared to longer measurements.\cite{mcnames2006reliability}
    Based on this, the research will focus on short-term HRV measurements, as it is more practical for the user, and it provides accurate 
    estimations. It will be more detailed in the methodology section.

    \textbf{Heart Rate Max:} is the maximum number of times the heart can beat in a minute, and it is a measure of the body's 
    cardiovascular fitness. It is an important indicator of the body's physical condition, and it is used to evaluate the body's ability 
    to perform physical activities, reflecting their cardiovascular fitness. In this research, understanding HRmax is vital for assessing 
    participants fitness and endurance. It could indicates how well players can handle stress and maintain their concentration over 
    extended periods of time. By analysing alongside other biometric data this research could uncover valuable insight into optmizing 
    players performance in a digital game scenario.

    \textbf{Heart Rate Average:} is the average number of times the heart beats in a minute, and it is a measure of the body's physical 
    activity. It is also an important indicator of the body's physical condition,     and it is another measure of the body's 
    cardiovascular fitness. It is used to evaluate the body's ability to perform physical activities, reflecting their cardiovascular 
    fitness.

    \textbf{Active Step:} Active steps, as a mesure of physical activity, is an important indicator of the body's physical condition. It is 
    used to evaluate the body's ability to perform physical activities, reflecting their cardiovascular fitness. Counting the numbers of 
    steps taken daily can help track overall physical activity, and it is a good indicator of the body physical condition. It is also used 
    to evaluate the body's ability to perform physical activities, reflecting their cardiovascular fitness.
    

    \textbf{Quality of Sleep:} are important indicators of the body's recovery and readiness for physical activity. Active sleep is a 
    measure of the body's physical activity during sleep, and it is an important indicator of the body's recovery and readiness for 
    physical activity.High-quality sleep, marked by sufficient duration and minimal disruptions, is fundamental for overall health 
    impacting mood, cognitive function.  A research that evaluated Sleep and performance in Eathletes \cite{BONNAR2019647} uderscores the 
    critical role of sleep in Eathletes performance in Esports, showing its impact on cognitive functions as a crucial factor for 
    competitives success. Adequate sleep improves information processing, visual motor functioning, attention, working memory, and other 
    functions essential for decision-making and reaction time. On the other hand, sleep depravation can significantly impair these 
    cognitive abilities, potentially to poor performance. For this research it proves the importance of sleep in the context of gaming 
    performance, and how it can be used to predict the player's readiness and tune their performance in a digital game scenario.

    \section{Classification Model}
    \section{Regression Model}

    
    
    

    

    
